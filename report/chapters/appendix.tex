\chapter{Appendix}

\section{Random Numbers} \label{sec:random}

The random numbers used in this simulation were generated using the Rust \verb|rand| crate \autocite{rustRustrandomRand2024}.

Random numbers in a computer system are never truly random, they are generated using a pseudo-random number generator that use a source of entropy to generate a sequence of numbers that appear random.
Entropy is a source of randomness that is used to seed the pseudo-random number generator \autocite{mertensEntropyPseudorandomnumberGenerators2004}.
The better the source of entropy, the more random the numbers will appear.

The \verb|rand| crate uses the \verb|SmallRng| function to generate pseudo-random numbers.
The crate also has support for generating cryptographically secure random numbers using other functions such as \verb|StdRng|.
Cryptographically generated numbers come at a performance cost, so they are not used in this simulation.


\section{Code Snippets} \label{sec:code}

\begin{lstlisting}[caption=Move Cell (Rust), label={lst:moveCell},captionpos=b]
pub fn move_cell(position: &mut (usize, usize), 
  grid: &mut Vec<Vec<f64>>, diagonal: bool) {

  // get available positions
  let available_positions = {
      if diagonal {
          crate::position::available_diagonal(*position, grid)
      } else {
          crate::position::available_normal(*position, grid)
      }
  };

  // randomly select a new position
  let random_position = rand::random::<usize>() % available_positions.len();
  *position = available_positions[random_position];
  grid[position.0][position.1] = 1.0;
}
\end{lstlisting}

\clearpage

\begin{lstlisting}[caption=Available Normal (Rust), label={lst:availableNormal},captionpos=b]
  pub fn available_normal(position: (usize, usize),
    grid: &Vec<Vec<f64>>) -> Vec<(usize, usize)> {

    let (x, y) = position;
    let mut positions = Vec::new();

    // check up down left right
    if x > 0 { positions.push((x - 1, y)); }
    if x < grid.len() - 1 { positions.push((x + 1, y)); }
    if y > 0 { positions.push((x, y - 1)); }
    if y < grid[0].len() - 1 { positions.push((x, y + 1));}

    positions
}
\end{lstlisting}


\begin{lstlisting}[caption=Available Diagonal (Rust), label={lst:availableDiagonal},captionpos=b]
pub fn available_diagonal(position: (usize, usize)
  grid: &Vec<Vec<f64>>) -> Vec<(usize, usize)> {

  let (x, y) = position;
  let mut positions = Vec::new();

  // check up down left right
  if x > 0 { positions.push((x - 1, y)); }
  if x < grid.len() - 1 { positions.push((x + 1, y)); }
  if y > 0 { positions.push((x, y - 1)); }
  if y < grid[0].len() - 1 { positions.push((x, y + 1)); }

  // check diagonals
  if x > 0 && y > 0 { positions.push((x - 1, y - 1)); }
  if x > 0 && y < grid[0].len() - 1 { positions.push((x - 1, y + 1)); }
  if x < grid.len() - 1 && y > 0 { positions.push((x + 1, y - 1)); }
  if x < grid.len() - 1 && y < grid[0].len() - 1 { 
    positions.push((x + 1, y + 1)); }

  positions
}
\end{lstlisting}

\clearpage

\begin{lstlisting}[caption=Growth Model (Rust), label={lst:growthModel},captionpos=b]
pub fn simulate(k: f64, m: f64, dt: f64, initial_n: f64,
  t_final: usize, condition: bool) -> Vec<f64> {
    
  // initialize an array to store the number of cells
  let mut n: Vec<f64> = vec![0.0; t_final];
  n[0] = initial_n;
  let mut t_current: usize = 0;

  // euler method to solve the differential equation
  for i in 1..t_final {
      t_current = i;

      // perform the euler method
      let dn_dt = k * n[i - 1] * (m / n[i - 1]).ln();
      n[i] = n[i - 1] + dn_dt * dt;

      // exit early if dn_dt is greater than 99.31% of m
      if n[i] >= 0.9931 * m && condition {
          break;
      }
  }
  
  // slice the array to the current time
  n.truncate(t_current);
  n
}
\end{lstlisting}


\clearpage

\section{Integration of Growth Model} \label{sec:substitution}

Integrate using substitution:

\begin{eqnarray*}
\frac{dN}{dt}                                &=& kNln\left(\frac{M}{N} \right) \\
\frac{dN}{Nln\left(\frac{M}{N} \right)}      &=& kdt \\
\int \frac{dN}{Nln\left(\frac{M}{N} \right)} &=& \int kdt \\
\end{eqnarray*}

\begin{align*}
\text{with}\ u &= ln\left(\frac{M}{N} \right) & \text{and}\ && \frac{du}{dN} &= -\frac{1}{N}
\end{align*}
% dN = -Ndu \]

\begin{eqnarray*}
\int \frac{-Ndu}{Nu} &=& \int kdt \\
\int -\frac{1}{u}du &=& \int kdt \\
-ln\left(|{u}| \right) &=& kt + c \\
ln\left(|{u}| \right) &=& -kt - c \\
ln\left(\left|\ln\left(\dfrac{M}{N}\right)\right|\right) &=& -kt - c \\
ln\left(\dfrac{M}{N}\right) &=& e^{-kt - c} \\
\dfrac{M}{N}                &=& e^{e^{-kt - c}} \\
N                           &=& \dfrac{M}{e^{e^{-kt - c}}}
\end{eqnarray*}


\clearpage

Calculate c using the initial values provided:

\begin{align*}
    M &= 10^{13} & k &= 0.06 & N &= 10^9 & t &= 0
\end{align*}

\begin{eqnarray*}
10^9        &=& \dfrac{10^{13}}{e^{e^{-0.006(0) - c}}} \\
e^{e^{-c}}  &=& \dfrac{10^{13}}{10^9} \\
e^{e^{-c}}  &=& {10^4} \\
-c          &=& ln\left(\left|\ln\left({10^4}\right)\right|\right) \\
c           &=& -ln\left(\left|\ln\left({10^4}\right)\right|\right) \\
\end{eqnarray*}

Substitute c back into the equation and simplify:

\begin{eqnarray*}
N &=& \dfrac{M}{e^{e^{-kt + ln\left(\left|\ln\left({10^4}\right)\right|\right)}}} \\
  &=& \dfrac{M}{e^{e^{-kt} e^{ln\left(\left|\ln\left({10^4}\right)\right|\right)}}} \\
  &=& \dfrac{M}{e^{e^{-kt} \ln\left({10^4}\right)}} \\
  &=& \dfrac{M}{(e^{\ln\left({10^4}\right)})^{e^{-kt}}} \\
N  &=& \dfrac{M}{10^{4e^{-kt}}}
\end{eqnarray*}

\clearpage