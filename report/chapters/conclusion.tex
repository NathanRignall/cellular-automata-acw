\chapter{Conclusion}

This report successfully implemented a simulation of the growth of cancer cells in a tissue using computational techniques.
The model's complexity is of $\bigO(n) + \bigO(n) = \bigO(n)$, where $n$ is the number of steps in the simulation.

Given a starting location any destination is equally likely, this is an artifact of the simulation.
Currently the model has no interaction between movement and growth.
A Bernoulli distribution for movement based on surface tension and leaky boundary conditions could be implemented to simulate the growth of cancer cells more accurately.

% The euler method was used 
The Euler method provided a simple and efficient way to solve the differential equation, with an accuracy of 99.95\% at $h = 0.01$.

Processing power was not a constraint, if this model was to be used in a low-power edge device, the simulation would need to be optimized.
Values for $h$ could be increased, movement directions could be reduced to 4, and the grid size could be reduced.

% Also talk about edge processing and limited requirements
% Issues with the numerical strategy include the accuracy of the simulation and the computation time.